\chapter{Groups}
When I started studying this subject, I though Abstract Algebra was a field about theoretical mathematics, which of course I was wrong.

Abstract Algebra, especially Groups, study the natural structure of life (or mathematical objects) in general (or in abstract) view.
Symmetries, Permutations, Rotations are part of the study of the Group Theory and many scientists (and mathematicians obviously) such as
modern Physicists and Chemists heavily use this subject.
\footnote{This is what I understood from stuff I have read in the internet}
\section{Introduction to Groups}
Group Theory is fairly new subject and through the history had subject to evolutionary process. Naturally it had many different definitions
and properties depending on these definitions. The modern definition is as follows,

\begin{definition*}
    Let $G$ be a set. A \alert{Binary Operation} on $G$ is a function that inputs a pair of elements in $G$ to another element of $G$.
\end{definition*}
\begin{example*}
    Addition and Multiplication are Binary Operations in $\ZZ$.
\end{example*}
\begin{definition*}
    A \alert{Group} $(G,\ast)$ is a set with a binary operation on $G$ that satisfies
    \begin{enumerate}
        \item Closure: $\forall x,y \in G$, $x \ast y \in G$.
        \item Associativity
        \item Identity (neutral): $\exists e \in G$ such that $x \ast e = x$ $\forall x \in G$.
        \item Inverse: $\forall x \in G$, $\exists x \inv \in G$ such that $x \ast x \inv = e$.
    \end{enumerate}
\end{definition*}
\begin{example*}
    There are several Groups we are already familiar with: $(\ZZ, +),(\QQ, +),(\RR, +)$ are all groups under addition. The identity element
    is $0$ and the inverses are simply $-a$ $ \forall a \in G$.
\end{example*}
\begin{example*}
    $(\QQ^+,*)$ is a Group with identity $1$. Clearly the inverse of $\frac{m}{n} \in G$ is $\frac{n}{m}$.
\end{example*}
\begin{example*}
    Addition of finite integers modulo $n$ or $(\ZZ_n, + \mod n)$ is a group. We denote this as $\ZZ_n$. For the specific case $n= 5$, we have
    \[ \{ 0,1,2,3,4\} \in \ZZ_n\]
    Identity element is $0$. We also call specific groups like these $\alert{Cyclic Groups}$. We will learn about them later.
\end{example*}
\begin{example*}
    A set of rotations and reflections of $n$-gon where $n \ge 3$ is a group with composition as operation. We call this $\alert{Dihedral
    Group}$ and denote this group as $D_n$.
    For $n =3$, $D_n$ consists of rotating $0^\circ, 120^ \circ ,240 ^\circ$ and reflecting about three bisects. There are total $6$
    elements of $D_3$.
\end{example*}
\section{Elementary Properties of Groups}
Now we will list very elementary properties and theorems of the groups with proofs (of course).
\begin{theorem}
    In the group $G$, identity is unique.
\end{theorem}
\begin{proof}
    Suppose otherwise. If both $e_1,e_2$ are distinct identity elements, then
    \[ e_1 = e_1 \ast e_2 = e_2\]
    Which is a clear contradiction.
\end{proof}
Group theory heavily studies the groups under binary operations $+$ and $\cdot $. In order to make notation more clear and easier, we use
(for multiplication)
\[ a \cdot b = ab, e = 1, a^n= aaa\ldots a\]
and for addition,
\[ a+b,e = 0, -a = a\inv, na= a + a + \ldots + a\]
Usually groups with addition operations are $\alert{commutative}$. Primary reason is mathematicians do not like seeing $a+b \neq b + a$. We
also have a specific name for such groups
\begin{definition*}
    A Group is $\alert{abelian}$ if it is commutative.
\end{definition*}
\begin{theorem}(Cancellation Law)
    in a group $G$, cancellation holds,
    \[ ba = ca \Rightarrow b = c \ \land  \ ab = ac \Rightarrow b = c\]
\end{theorem}
\begin{proof}
    Multiply by $a\inv$ from right and left respectively.
\end{proof}
\begin{theorem}
    In the group $G$, inverses are unique.
\end{theorem}
\begin{proof}
    Suppose otherwise. If both $b$ and $c$ are inverses of $a$, then
    \[ a \ast b  = a \ast c = e\]
    Cancelling $a$ gives $b \inv = c \inv$
\end{proof}
\begin{theorem}
    For all $a,b$ elements of a group,
    \footnote{Notice how this theorem looks very familiar for Linear Algebra's inverse theorem. No surprise here, invertible matrices under
    multiplication is a group.}
    \[ (ab)\inv = b \inv a \inv\]

\end{theorem}

\begin{proof}
    We have
    \[ (ab) \inv \ast ab = e = b\inv a \inv ab  = (b \inv a \inv) (ab)\]
\end{proof}
